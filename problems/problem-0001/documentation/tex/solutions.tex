%% tex/description.tex

\section*{Solution(s)}

\subsection*{Solution 1: Brute Force}

The brute force approach here is an $O(n^2)$ algorithm using nested \code{for} loops.

\vspace*{1em}
\begin{algorithm}[H]
    \SetAlgoLined
    \KwData{A sequence $(a_i)_{i \in [0, n)}$ of $n$ integers}
    \KwData{An integer $N$}
    \KwResult{The unique unordered pair $\{j, k\}$ of the indices $j, k \in [0, n)$ of the two numbers from $(a_i)_{i \in [0, n)}$ s.t.~$a_j + a_k = N$}
    \BlankLine
    \For{$j \in [0, n)$}{
        \For{$k \in (j, n)$} {
            \If{$a_j + a_k = N$} {
                \Return{$\{j, k\}$}
            }
        }
    }
\end{algorithm}
\vspace*{1em}

The worst case scenario for this algorithm is when the numbers are the last two in the array, in which case the total number of iterations is
$$
(n - 1) + (n - 2) + \cdots + 2 + 1 = \sum_{i=1}^{n-1} i = \frac{n(n - 1)}{2} - n = \frac{1}{2}n^2 - \frac{3}{2}n.
$$
Therefore, the algorithm has, as stated above, $O(n^2)$ time complexity.

The benefit of this algorithm is that it is very easy to understand and to implement.


\subsection*{Solution 2}

A more efficient approach is as follows:

\vspace*{1em}
\begin{algorithm}[H]
    \SetAlgoLined
    \KwData{A sequence $(a_i)_{i \in [0, n)}$ of $n$ integers}
    \KwData{An integer $N$}
    \KwResult{The unique unordered pair $\{j, k\}$ of the indices $j, k \in [0, n)$ of the two numbers from $(a_i)_{i \in [0, n)}$ s.t.~$a_j + a_k = N$}
    \BlankLine
    $X_0 \gets \{(a_0, 0)\}$\;
    \For{$j \in (0, n)$}{
        $x \gets N - a_j$\;
        \eIf{$x \in \pi_{j1}(X_j)$} {
            \Return{$\{\pi_{j2}(x), j\}$}
        }{
            $X_j \gets X_{j - 1} \bigcup \{(a_j, j)\}$\;
        }
    }
\end{algorithm}
\vspace*{1em}

The idea behind the sets $X_0, X_1, \ldots, X_{n - 1}$ is to keep track of the elements of $(a_i)_{i \in [0, n)}$ that have been ``visited'' during the iteration process, whose elements are the ordered pairs of the elements $a_0, a_1, \ldots, a_{n - 1} \in (a_i)_{i \in [0, n)}$ and their associated indices, so that, for each $k \in [0, n)$, $X_k = \{(a_0, 0), (a_1, 1), \ldots, (a_k, k)\}$. For each $k \in [0, n)$, the functions $\pi_{k \alpha}$ are the projection functions on $X_k$ which map each $(a_k, k)$ to the $\alpha^{\text{th}}$ coordinate:
\begin{align*}
  \pi_{k1} &: (a_k, k) \mapsto a_k \\
  \pi_{k2} &: (a_k, k) \mapsto k.
\end{align*}
At the $k^{\text{th}}$ iteration, the idea is to search for a number $x \in (a_i)_{i \in [0, n)}$ that satisfies $x + a_k = N$, or, equivalently, search for an $x \in (a_i)_{i \in [0, n)}$ that satisfies $x = N - a_k$.

The worst case scenario for this algorithm is when the numbers are the last two in the array, in which case $n - 1$ iterations would be required. If the search operation can be performed in $O(1)$ time, this algorithm is therefore $O(n)$.


\clearpage
