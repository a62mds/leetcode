%% tex/description.tex

\section*{Solution(s)}

\subsection*{Solution 1: Brute Force}

The brute force approach here is an $O(n^2)$ algorithm using nested \code{for} loops.

\vspace*{1em}
\begin{algorithm}[H]
    \SetAlgoLined
    \KwData{A sequence $(a_i)_{i \in [0, n)}$ of $n$ integers}
    \KwData{An integer $N$}
    \KwResult{The unique unordered pair $\{j, k\}$ of the indices $j, k \in [0, n)$ of the two numbers from $(a_i)_{i \in [0, n)}$ s.t.~$a_j + a_k = N$}
    \BlankLine
    \For{$j \in [0, n)$}{
        \For{$k \in (j, n)$} {
            \If{$a_j + a_k = N$} {
                \Return{$\{j, k\}$}
            }
        }
    }
\end{algorithm}
\vspace*{1em}

The worst case scenario for this algorithm is when the numbers are the last two in the array, in which case the total number of iterations is
$$
(n - 1) + (n - 2) + \cdots + 2 + 1 = \sum_{i=1}^{n-1} i = \frac{n(n - 1)}{2} - n = \frac{1}{2}n^2 - \frac{3}{2}n.
$$
Therefore, the algorithm has, as stated above, $O(n^2)$ time complexity.

The major benefit of this algorithm is that it is very easy to understand and to implement. A second benefit is that it has $O(1)$ space complexity.


\subsection*{Solution 2}

A more efficient approach is as follows:

\vspace*{1em}
\begin{algorithm}[H]
    \SetAlgoLined
    \KwData{A sequence $(a_i)_{i \in [0, n)}$ of $n$ integers}
    \KwData{An integer $N$}
    \KwResult{The unique unordered pair $\{j, k\}$ of the indices $j, k \in [0, n)$, $j \neq k$, of the two numbers from $(a_i)_{i \in [0, n)}$ s.t.~$a_j + a_k = N$}
    \BlankLine
    $X \gets \varnothing$\;
    \For{$i \in [0, n)$}{
        $x \gets N - a_i$\;
        \eIf{$x \in \pi_1(X)$} {
            \Return{$\{\pi_2\circ\pi_1^{-1}(x), i\}$}
        }{
            $X \gets X \bigcup \{(a_i, i)\}$\;
        }
    }
\end{algorithm}
\vspace*{1em}

The idea behind the set $X$ is to keep track of the elements of $(a_i)_{i \in [0, n)}$ that have been ``visited'' during the iteration process. At the end of the $k^{\text{th}}$ iteration, $k \in [0, n)$, the elements of $X$ are just the ordered pairs $(a_0, 0), (a_1, 1), \ldots, (a_k, k)$ of the elements $a_0, a_1, \ldots, a_k \in (a_i)_{i \in [0, n)}$ and their associated indices. The functions $\pi_{\alpha}$ are the projection functions on $X$ which map each $(a_k, k)$ to the $\alpha^{\text{th}}$ coordinate:
\begin{align*}
  \pi_1 &: (x, i_x) \mapsto x \\
  \pi_2 &: (x, i_x) \mapsto i_x.
\end{align*}
There is a bit of a technical issue with the notion of the inverse $\pi_1^{-1}$ of the first projection function $\pi_1$ in that it's possible for there to exist $j, k \in [0, n)$, $j \neq k$, that satisfy $a_j = a_k$. This would mean that, if $x = a_j = a_k$, $\pi^{-1}$ would map $x$ to both $j$ and $k$, violating the definition of a function. However, the constraint that there is only one valid answer ensures that if an $x$ exists that satisfies $x = N - a_i$ for some $i \in [0, n)$, it is unique. A restriction on the range of $\pi_1$ could be made to make the definition rigorous, however just allowing the abuse of notation with the above understanding seems like the simpler and clearer choice here.

At the $k^{\text{th}}$ iteration, the idea is to search $X$ for a the pair $(x, i_x) \in X$ that satisfies $x + a_k = N$, or, equivalently, $x = N - a_k$. If no such $x$ exists, the set $X$ gets expanded by adding $(a_k, k)$, which in essence marks $a_k$ as having been ``visited''. The worst case scenario for this algorithm is when $N - a_{n - 1}$ gives the first match, in which case $n$ iterations are required. If the search operation can be performed in $O(1)$ time, the time complexity of this algorithm is therefore $O(n)$. Also in this worst case, the set $x$ will contain $n - 1$ elements, so the space complexity is $O(n)$.


\clearpage
